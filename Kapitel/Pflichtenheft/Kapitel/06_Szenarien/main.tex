\section{Szenarien}

\begin{frame}\frametitle{Szenarien}
Szenario Nr. 5: Interagieren mit der Etagenkarte vom Gebäude 50.34
\begin{itemize}
	\item \textbf{Anfangssituation:} Vereinbarung von Terminen in Büroräumen mit KIT-Angestellten
\end{itemize}
\begin{tabular}{cl}
	\begin{tabular}{c}
		\includegraphics[scale=0.07]{\relimgfile{student.png}}
	\end{tabular}
	& \parbox{11cm}{\begin{itemize}
		\item bekannt
		\begin{itemize}
			\item Namen der KIT-Angestellten, mit denen man sich trifft \\ (Prof. Dr. Ralf H. Reussner, M.Sc. Erik Burger)
			\item beide Büroräume im Gebäude 50.34
			\item Büroraum von M.Sc. Erik Burger im 2. OG
		\end{itemize}
		\item unbekannt
		\begin{itemize}
			\item um welche Büroräume handelt es sich genau? (Raumnummer)
		\end{itemize}
	\end{itemize}} \\
\end{tabular}
\end{frame}
\begin{frame}
\begin{itemize}\frametitle{Szenarien}
	\item \textbf{Handlung}
\end{itemize}
	\smartdiagramset{back arrow disabled=true,text width=2cm, font=\fontsize{6pt}{12pt}\selectfont}
	\smartdiagram[flow diagram:horizontal]{% 
    		App öffnen\\
    		$\Rightarrow$ Kartenansicht,
    		Suchfunktion: \texttt{Prof. Dr. Ralf H. Reussner} eingeben \\
    		Eingabe bestätigen,
    		App findet Raum 327\text{,} Gebäude 50.34 \\
    		zugleich zeigt App die Etagenkartenansicht \\
   		d.h. Stockwerk der gezeigten Etagenkarte\text{,} 3. OG,
    		nun bekannt: \\
    		Büroraum von Prof. Dr. Ralf H. Reussner
	}
\end{frame}
\begin{frame}\frametitle{Szenarien}
\smartdiagramset{back arrow disabled=true,text width=2cm, font=\fontsize{6pt}{12pt}\selectfont}
	\smartdiagram[flow diagram:horizontal]{%
	Etagenkartenansicht: \\
   	Etage wechseln (zum 2. OG),
    	Anschauen von Infos von einigen Räumen vom 2. OG \\
    	$\Rightarrow$ Raum 239 \\
    	App zeigt an\text{,} dass Raum 239 der Büroraum von M.Sc. Erik Burger ist,
    	Etagenkartenansicht: \\
    	dadurch Treppe lokalisieren\text{,} die zu den Etagen führt,
    	man macht sich auf dem Weg zu den Büroräumen
    	}
\end{frame}
