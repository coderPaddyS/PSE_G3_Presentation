\section{KIT-Anmeldung}

\begin{frame}\frametitle{Keine eigene Anmeldung}
    \begin{itemize}
        \item Warum? \pause
        \item Jeder kann sich einen Account machen. Wie verhindern wir Spam? KIT-Datenschutz?
        \item Nutzerdaten müssen sicher gespeichert werden: \begin{itemize}
            \item Passwörter müssen gehashed werden.
            \item Gleiche Passwörter bei unterschiedlichen Benutzern müssen anders gehashed werden.
            \item Sicherheitskriterien für Passwörter müssen durchgesetzt werden.
            \item Datenübertragung muss verschlüsselt sein.
            \item Zugangsdaten für die DB müssen gesichert gelagert sein.
        \end{itemize}
        \item Wie kann man das Passwort erneuern? Was passiert wenn das Passwort vergessen wurde?
        \item Wie sichern wir den Server gegen Angriffe?
    \end{itemize}
\end{frame}

\begin{frame}\frametitle{Anmeldung mit Shibboleth via OIDC}
    \begin{itemize}
        \item Vorherige Probleme werden ans KIT delegiert.
        \item Anmeldung mit KIT-Account.
        \item Einfachere Schnittstelle mit Shibboleth.
        \item Neuester Sicherheitsstandard dank OIDC.
        \item Token-Basierte Authentifizierung, verschlüsselte Nutzerdaten direkt enthalten.
        \pause
        \item Nachteil: Tokens laufen nach 30 min Inaktivität ab
        \item[$\rightarrow$] Neuauthentifizierung nötig.
    \end{itemize}
\end{frame}

\begin{frame}\frametitle{Typischer Ablauf eines OIDC-Code-Flows}
    \centering
    \inprelimg[height=0.8\textheight]{oidc_auth_code}
\end{frame}

\begin{frame}\frametitle{Hybride Anmeldung}
    \begin{itemize}
        \item Geplant: Hybride Anmeldung
        \item Registrierung über Shibboleth.
        \item Erstellung eigener, verschlüsselter Tokens durchs Backend, länger gültig.
        \item Übertragung der Nutzerdaten in den Token $\rightarrow$ Keine Nutzerdaten zu speichern.
        \item Neuauthentifizierung spätestens nach einem Semester um Studentenstatus zu validieren.
    \end{itemize}
\end{frame}