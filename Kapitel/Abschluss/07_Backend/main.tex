\section{Backend}

\begin{frame}\frametitle{Architektur}
    \begin{itemize}
        \item Tools
        	\begin{itemize}
        		\item Spring Boot
        		\item GraphQL
        		\begin{itemize}
        			\item Open-Source-Datenabfrage- und Manipulationssprache für die API
        			\item Alternative zu REST (ermöglicht Kommunikation zwischen Frontend und Backend)
        		\end{itemize}
        	\end{itemize}
        	\item hauptsächliche Struktur des Backends: 3-Schichten-Architektur
      	\begin{tabular}{cl}
    			\inprelimg[width=.6\textwidth]{packages.png}
    		\end{tabular}
    \end{itemize}
\end{frame}

\begin{frame}\frametitle{Erfahrungen mit GraphQL}
    \begin{itemize}
        \item Vergleich zu REST
        		\begin{itemize}
        			\item Nur benötigte Daten werden Abgefragt
        			\item Empfänger entscheidet, welche Daten er benötigt 
        			\item Schema an einer Stelle erstellt und verwaltet
        			\item Alle Abfragen im Schema präzise definiert
        			\item Sicherheit schwieriger umsetzbar
        		\end{itemize}
        \item Fazit: Komplizierter zu verstehen/erlernen als REST,
				dafür aber deutlich schnellere/effizientere Entwicklung und Nutzung
    \end{itemize}
\end{frame}