\section{Team-Koordination}

\begin{frame}\frametitle{Koordination im Team}
    \begin{itemize}
        \item größtenteils in Ordnung
    \end{itemize}
        	\begin{tabular}{l l}
        	\begin{tcolorbox}[width=.4\textwidth]
        	Planungsphase
        	\begin{itemize}
        		\item optimale Arbeitsaufteilung
        		\item alle hatten am Ende der Phase einen guten Durchblick
        		\item Meetings mehrmals in der Woche $\Rightarrow$ alle sind auf dem Laufenden
        	\end{itemize}
        	\end{tcolorbox} &
        	\begin{tcolorbox}[width=.5\textwidth]
        	Entwurfsphase
        	\begin{itemize}
        		\item Aufteilung in Teams
        		\begin{itemize}
        			\item Admin-Panel: 1 Person
        			\item App \& Backend: jeweils 2 Personen
        		\end{itemize}
        	\item weiterhin Meetings wie schon in der Planungsphase $\Rightarrow$ teamübergreifende Kommunikation
     	\end{itemize}
        	\end{tcolorbox} \\
        	\end{tabular}
\end{frame}

\begin{frame}\frametitle{Koordination im Team}
   	\begin{tcolorbox}[width=.9\textwidth]
   	Implementierungsphase
   		\begin{itemize}
   		\item Unterschätzung des Aufwands der App
   		\begin{itemize}
   			\item Plan: Verantwortliche des Admin-Panels und Backends kommen später zur App hinzu
   		\end{itemize}
   		\item trotz regulären Meetings hatten Admin-Panel und Backend nicht den besten Durchblick des Codes der App $\Rightarrow$ Einarbeitungszeit und detailreiche Anweisungen von der App notwendig
   	\end{itemize}
    \end{tcolorbox}
    Fazit
    \begin{itemize}
        	\item Fokus in den eigenen Bereichen wichtig $\Rightarrow$ Teamaufteilung auch zurückblickend sinnvoll
        	\item besser: 1 Person alleine für das Backend zuständig $\Rightarrow$ 3 Personen widmen sich vollzeitig der App
    \end{itemize}
\end{frame}
